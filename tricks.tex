\chapter{Triks}
Ada beberapa hal menarik dari Python. Pada bagian ini akan ditampilkan berbagai
trik tersebut.

\section{Server Web}
Python dapat kita gunakan sebagai server web sederhana yang dapat memerikan
list direktori. Fitur ini sering saya gunakan untuk transfer berkas antar
mesin. Misalnya saya ingin mengambil sebuah berkas dari satu komputer ke
komputer yang lain. Maka komputer dimana berkas tersebut berada saya jalankan
perintah ini (pada direktori dimana berkas tersebut berada).

\begin{verbatim}
python -m SimpleHTTPServer 8008
\end{verbatim}

Perintah di atas akan membuat sebuah server web yang berjalan pada {\em port}
8008. Anda dapat menggunakan port yang berbeda.
Akan ditampilkan daftar berkas yang ada di direktori tersebut. Saya tinggal
memilih berkas yang dimaksudkan dan {\em Save as ...}.

Fitur server web ini juga sering saya gunakan untuk proses {\em debugging}
sebuah aplikasi yang menggunakan protokol web. Untuk melihat data yang
dikirimkan oleh aplikasi tersebut, saya arahkan dia ke web server ini dan
kemudian saya perhatikan apa yang ditampilkan (yang diminta oleh aplikasi
tersebut).

\section{Membaca Berkas Konfugurasi config.ini}
Seringkali ada kebutuhan untuk menyimpan {\em credential} (userid, password)
dalam aplikasi. Sebagai contoh, kita ingin membuat koneksi ke server lain
(server database, server FTP, dan sejenisnya) maka dibutuhkan informasi
tentang nama atau nomor IP dari server yang dituju, userid dan password,
dan informasi lainnya lagi.
Seringkali informasi ini ditanam di dalam kode (source code).

Apa beberapa masalah terkait dengan pendekatan ini (menanam credential)
di dalam kode sumber.
Pertama, jika terjadi perubahan (misal ada pergantian nomor IP dari server
yang dituju) maka kita harus mencari di dalam kode mana saja terdapat
penanaman informasi tersebut dan tentu saja harus diperbaharui.
Bayangkan kalau informasi ini terdapat di beberapa berkas. 
Bagaimana kita tahu itu ada dimana?
Jika konfigurasi ini hanya ada di satu berkas, misalnya di berkas
``config.ini'', maka operator hanya perlu melakukan perubahaan di berkas ini.

Kedua, perubahan terhadap kode sangat berisiko. Beberapa kali terjadi
kasus seseorang mengubah kode dari sebuah aplikasi (dengan bayangan
minor update), ternyata aplikasi menjadi tidak berjalan.
Ini dapat terjadi ketika update kekurangan tanda petik, misalnya,
sehingga aplikasi malah menjadi ``error''.

Jika kode ingin dibagikan, misal menggunakan github, maka credential
sudah terpisahkan dengan kode sehingga kemungkinan tersebar juga lebih kecil.
Tentu saja masih mungkin terjadi jika berkas config.ini berada dalam
direktori yang dibagikan (share) dan tidak dimasukkan ke dalam
berkas (.gitignore) dalam kasus github.

Berikut ini adalah contoh kode yang lazim ditemui, yaitu menanamkan
{\em credential} di dalam kode sumber. Perhatikan bahwa alamat server,
userid, dan password tertanam dalam kode.

\begin{verbatim}
from ftplib import FTP

SERVER = '192.168.4.29'
NAMA = 'jabar'
GEMBOK = 'juara2020'

ftp = FTP(SERVER)
ftp.login(user=NAMA, passwd=GEMBOK)
ftp.dir()
ftp.quit()
\end{verbatim}

Cara yang lebih baik adalah dengan menyimpan credential tersebut
ke dalam sebuah berkas konfigurasi. Berikut ini adalah sebuah
contoh isi berkas ``config.ini''.

\begin{verbatim}
[aplikasiku]
   server = 192.168.4.29
   user = jabar
   password = juara2020
[queue]
   server = rabbitmq-server
   user = rabbitku
   password = sangatrahasia6677
\end{verbatim}

Berkas ini dapat dibaca dari aplikasi dengan cara berikut, dengan asumsi
bahwa berkas ``config.ini'' berada dalam direktori yang sama dengan
aplikasi ini. Untuk lebih meningkatkan keamanan, seringkali letak berkas
konfigurasi berada di luar struktur direktori ini (di atasnya). Jadi
variabel CONF dapat diisi dengan ``../config.in''.


\begin{verbatim}
from ftplib import FTP
from configparser import ConfigParser

CONF='config.ini'
config = ConfigParser()
# return filename
config.read(CONF)

SERVER = config.get('aplikasiku','server')
NAMA = config.get('aplikasiku','user')
GEMBOK = config.get('aplikasiku','password')

ftp = FTP(SERVER)
ftp.login(user=NAMA, passwd=GEMBOK)
ftp.dir()
ftp.quit()
\end{verbatim}

Penjelasan mengenai ini ada dalam video berikut.
\begin{verbatim}
https://www.youtube.com/watch?v=jze3DC6ohRc
\end{verbatim}
