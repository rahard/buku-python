\chapter{Pendahuluan}
Bahasa pemrograman Python mulai populer saat dikarenakan berbagai hal; mudah
dipelajari, tersedia dan banyak {\em library}-nya. Nanti akan kita bahas beberapa
library Python ini. Lengkapnya library ini juga yang menyebabkan
Python dipergunakan di berbagai aplikasi. Berbagai sekolah (dan
perguruan tinggi) bahkan mengajarkan Python sebagai pengantar pemrograman.

Bahasa Python merupakan sebuah {\em interpreted language} berbeda dengan
bahasa C yang {\em compiled}. Pada bahasa yang {\em compiled}, kita memiliki
kode sumber ({\em source code}) yang harus dirakit ({\em compile}) dahulu
sampai menjadi kode mesin yang langsung dapat dieksekusi pada komputer
yang bersangkutan. Ketika algoritma salah, maka kode sumber harus diperbaiki
dahulu kemudian di-{\em compile}) sebelum dapat dijalankan. Prosesnya
menjadi agak panjang. Sementara itu untuk bahasa yang {\em interpreted},
program langsung dieksekusi dari kode sumbernya (tanpa perlu proses kompilasi).
Dahulu program yang {\em compiled} lebih cepat dalam eksekusinya karena
tidak perlu menerjemahkan baris perbaris ketika dijalankan, namun sekarang
perbedaannya sudah tipis.

Bahasa Python tersedia untuk berbagai sistem operasi; Windows, Mac OS, dan
berbagai variasi dari UNIX (Linux, *BSD, dan seterusnya). Di dalam buku ini
saya akan menggunakan contoh-contoh yang saya gunakan di komputer saya yang
berbasis Linux Mint. Meskipun seharusnya semuanya kompatibel dengan berbagai
sistem operasi, kemungkinan ada hal-hal yang agak berbeda. Jika hal itu
terjadi, gunakan internet untuk mencari jawabannya.

\section{Instalasi}
Python dapat diperoleh secara gratis dari berbagai sumber. Sumber utamanya
adalah di situs python.org. Untuk sementara ini bagian ini saya serahkan kepada
Anda. Ada terlalu banyak perubahan sehingga bagian ini akan cepat kadaluwarsa.
Untuk sistem berbasis sistem operasi Microsoft Windows, biasanya instalasi 
Python menggungankan {\em Anaconda}. (Informasi mengenai ini juga dapat
dilihat pada situs python.org.)

Untuk sistem operasi berbasis Linux dan Mac OS, Python sudah terpasang sebagai
bawaan dari sistem operasinya. Jika Anda ingin memasang versi terbaru maka
Anda harus memasangnya sendiri dengan mengunduh instalasinya di python.org.
Atau, jika Python sudah terpasang di komputer Anda, maka Anda dapat melakukan
{\em upgrade}.

\section{Memulai}
Untuk memastikan Python berjalan, ketikkan "python" di terminal Linux Anda.
(Bagi yang menggunakan Windows, hal ini dapat dilakukan dengan menggunakan
CMD.exe.) Catatan, di sistem Linux, tanda ``dollar'' merupakan {\em prompt}
dari {\em shell} Anda. Jangan diketikkan.

\begin{verbatim}
$ python
Python 2.7.12 (default, Nov 20 2017, 18:23:56) 
[GCC 5.4.0 20160609] on linux2
Type "help", "copyright", "credits" or "license" for more information.
>>> 
\end{verbatim}

Dari tampilan di atas dapat kita ketahui bahwa Python yang saya gunakan adalah
versi 2.7.12. Sekarang kita dapat memulai pemrograman Python dengan menuliskan
program ``hello world'' (yang merupakan standar bagi belajar pemrograman).
Ketikkan ``print ...'' (dan seterusnya seperti di bawah ini).

\begin{verbatim}
print "Hello, world!"
Hello, world!
\end{verbatim}

Python akan menampilkan apapun yang ada di antara tanda petik tersebut. Hore!
Anda berhasil membuat program Python yang pertama.

Ada beberapa cara untuk menjalankan Python. Pada contoh di atas, kita
menjalankannya secara langsung. Cara ini memang yang paling cepat, tetapi ada
banyak hal yang harus kita lakukan secara manual. Sebagai contoh, jika kita
ingin membuat sebuah {\em block}, maka kita harus mengetikkan sendiri empat
spasi untuk membuatnya masuk. Jika ini tidak kita lakukan, maka dia akan
``marah'' dan menampilkan pesan. Berikut ini contoh sesi yang salah.

\begin{verbatim}
$ python3
Python 3.5.2 (default, Nov 23 2017, 16:37:01) 
[GCC 5.4.0 20160609] on linux
Type "help", "copyright", "credits" or "license" for more information.
>>> for i in range(10):
... print(i)
  File "<stdin>", line 2
    print(i)
        ^
IndentationError: expected an indented block
>>> 
\end{verbatim}

Pada contoh di atas, kesalahan terjadi karena kita tidak memberikan spasi di
depan perintah ``print(i)''. Seharusnya kita melakukan hal seperti ini.
(Perhatikan spasi sebelum kata ``print''.)

\begin{verbatim}
$ python3
Python 3.5.2 (default, Nov 23 2017, 16:37:01) 
[GCC 5.4.0 20160609] on linux
Type "help", "copyright", "credits" or "license" for more information.
>>> for i in range(10):
...     print(i)
... 
0
1
2
3
4
5
6
7
8
9
>>> 
\end{verbatim}


Mari kita lanjutkan dengan membuat program yang lebih panjang. Program Python
dapat disimpan di dalam sebuah berkas untuk kemudian dieksekusi belakangan.
Buka editor kesukaan Anda dan ketikkan program hello world di atas di dalam
editor Anda tersebut. Setelah itu simpan berkas tersebut dengan nama
``hello.py''. Biasanya berkas program Python ditandai dengan akhiran
(extension) ``.py''.

Setelah berkas tersebut tersedia, maka kita dapat menjalankan Python dengan
memberikan perintah python dan nama berkas tersebut. 
Kata ``Hello world'' akan ditampilkan.

\begin{verbatim}
$ python hello.py
Hello, world!
\end{verbatim}

Cara lain yang dapat dilakukan adalah dengan menggunakan program IPython,
yaitu interactive Python. Biasanya program IPython ini belum terpasang di
sistem operasi bawaan komputer Anda. Tinggal unduh dari ipython.org.
(Perhatikan IPython ini sudah mengenal {\em syntax} dari Python sehingga 
ketika kita mengetikkan {\em for loop} maka dia akan memberikan spasi 4 buah
sebagai {\em indentation}.)

\begin{verbatim}
$ ipython3
In [1]: print("Hello World")
Hello World

In [2]: for i in range(10):
   ...:     print(i)
   ...:     
0
1
2
3
4
5
6
7
8
9

In [3]: 
\end{verbatim}

Masih ada satu cara lain menjalankan program Python, yaitu dengan menggunakan
Jupyter. Yang ini akan kita bahas secara lebih khusus.

Pada contoh-contoh di atas hasil print dicetak ke bawah. Bagaimana jika kita
ingin hasil cetaknya tidak turun ke bawah atau tanpa {\em newline}?
Cara berikut ini - dengan menambahkan (end = " ") - dapat digunakan:

\begin{verbatim}
for n in range(10):
   print(n, end=" ")
\end{verbatim}
Keluaran dari program di atas adalah cetakan yang ke kanan.
\begin{verbatim}
0 1 2 3 4 5 6 7 8 9
\end{verbatim}


\section{Bahasa Python} 
Tentang bahasa Python itu sendiri akan diperdalam pada versi berikutnya.
Sementara itu fitur tentang bahasa Python akan dibahas sambil berjalan.
Pendekatan ini saya ambil untuk membuat buku menjadi lebih menarik dan lebih
singkat. Belajar seperlunya. Mari kita mulai.

Variabel di dalam Python langsung dapat digunakan tanpa melakukan deklarasi
sebelumnya. Pada bahasa pemrograman seperti C, variabel harus dideklarasikan
tipenya; apakah dia {\em integer} atau {\em string}.
Contohnya di bawah ini.
\begin{verbatim}
a = 7
b = 5
print(a,b)
\end{verbatim}
Pada contoh di atas, variabel $a$ dan $b$ dibuat dan langsung diisi dengan
angka (7 dan 5) dalam kasus ini.
Kemudian kedua variabel tersebut disampaikan sekaligus.
Perhatikan bahwa dengan menggunakan tanda koma (,) nilai dari kedua variabel
tersebut ditampilkan dengan spasi.

\begin{verbatim}
7 5
\end{verbatim}

Berikut ini kita buat penampilan yang lebih ``menarik'' (untuk Python3).
\begin{verbatim}
a = 7
b = 5
c = a + b
print ("a = ", a)
print ("b = ", b)
print ("a+b = ", c)
\end{verbatim}

Keluaran dari Python3 adalah seperti ini:
\begin{verbatim}
a = 7
b = 5
a+b = 12
\end{verbatim}

Hal yang sangat berbeda dari bahasa Python dengan
bahasa pemrograman lainnya adalah masalah {\em block} dari kode. 
Bahasa pemrograman C misalnya menggunakan tanda kurung kurawal ``\{'' 
untuk menyatakan blok. Sementara itu Python menggunakan {\em indentation} 
untuk menyatakan satu blok. Lihat contoh di bawah ini.

\begin{verbatim}
for i in range(5):
    for j in range(3):
        print(i,j)
\end{verbatim}

Disarankan untuk menggunakan spasi sebanyak empat (4) buah untuk {\em
indentation} tersebut. (Ini membuat banyak perdebatan karena ada banyak orang
yang menggunakan tab bukan spasi.)

Mari kita buat contoh-contoh lain. Apa keluaran program di bawah ini?
\begin{verbatim}
nama1 = "budi"
nama2 = "rahardjo"
nama3 = nama1 + nama2
print(nama3)
\end{verbatim}

Perhatikan bahwa untuk variabel yang berjenis angka (integer) maka operator 
tambah (+) akan menambahkan angkanya. Sementara itu pada variabel yang
berjenis {\em string}, operator (+) akan menyambungkan ({\em concatenate})
variabel {\em string} tersebut.
Ini merupakan ciri dari sebuah bahasa yang berorientasi obyek
({\em object oriented}).
Untuk tipe data yang lainnya, operator tambah (+) kemungkinan juga akan
memiliki perilaku yang berbeda.

Mari kembali ke data yang berbentuk angka. Apa keluaran dari program
di bawah ini?
\begin{verbatim}
a = 7
b = 5
c = a/b
print(c)
\end{verbatim}

Keluarannya adalah nilai ``1.4''. Jenis atau tipe data dari ``a'' dan ``c''
adalah berbeda. Tipe data ``a'' adalah {\em integer}, sementara itu
tipe data ``c'' adalah {\em floating point}.

\begin{verbatim}
>>> a=7
>>> type(a)
<class 'int'>
>>> b=5
>>> c=a/b
>>> type(c)
<class 'float'>
\end{verbatim}

Bolehkah kita mencampurkan tipe data yang berbeda?
Mari kita lanjutkan dengan contoh di atas.
\begin{verbatim}
d = a+c
print(d)
\end{verbatim}
Keluaran dari kode di atas adalah ``8.4''. Artinya pencampuran tipe data
yang berbeda (dalam hal ini adalah {\em integer} dan {\em float}) dapat
dilakukan. Hasilnya adalah bilangan {\em float}.
Hal ini menunjukkan bahwa bahasa Python dapat dibuat tidak {\em strict}.
Bahasa pemrograman lain - misalnya Java (dan umumnya bahasa pemrograman
yang berorientasi obyek) - sangat ketat dalam hal ini.
    
\section{Input}
Salah satu cara untuk mendapatkan masukan (input) dari pengguna secara
interaktif adalah dengan menggunakan fungsi "input" (untuk Python 3.*)
atau "raw\_input" (untuk Python 2.7).

\begin{verbatim}
# ini untuk Python 2.7
# gunakan raw_input
nama = raw_input("Masukkan nama Anda: ")
print "Selamat pagi,", nama
\end{verbatim}

Perhatikan bahwa kita menggunakan variabel "nama" untuk menyimpan masukan
dari pengguna. Variabel "nama" tersebut mempunyai tipe {\em string}.
Python mengenali secara otomatis.

Mari kita coba tampilkan huruf-huruf yang ada di dalam variabel "nama"
tersebut.

\begin{verbatim}
# for loop bisa menggunakan elemen dari string
# tidak harus indeks angka
for i in nama:
    print i
\end{verbatim}

Kita juga dapat membuat statistik kemunculan huruf dari nama (atau teks)
yang dimasukkan oleh pengguna. Statistik ini dapat dimanfaatkan untuk
proses enkripsi, misalnya. Gunakan program "input" di atas, dan gabungkan
dengan kode berikut ini.

\begin{verbatim}
# associative array: hitung jumlah huruf dan spasi
huruf = {}  # inisialisasi
for key in nama:
    if key in huruf:
        huruf[key] += 1
    else:
        huruf[key]=1
# tampilkan hasil python 2.7
# sorted() agar key-nya diurutkan
# for python 3.* use this: for key, value in d.items():
for key, value in sorted(huruf.iteritems()):
    print key, value
\end{verbatim}

Contoh program di atas menggunakan {\em associative array} atau dalam
Python disebut {\em dictionary}. Pada prinsipnya ini adalah array tetapi
dengan menggunakan {\em immutable object} seperti {\em string} sebagai
indeks atau kuncinya.

Pada contoh tersebut, spasi ({\em space}) masih dianggap sebagai huruf.
Coba ubah sehingga spasi tidak dimasukkan sebagai indeks.


\section{Pemrosesan Teks}
Salah satu manfaat utama dari bahasa pemrograman seperti Python adalah
kemampuannya dalam memproses teks ({\em text processing}). Bahasa
pemrograman lainnya, seperti C, tentu saja dapat digunakan untuk melakukan
pemrosesan teks. Namun bahasa C lebih "sulit" digunakan karena ada banyak
hal yang harus kita ketahui dari awal.

\begin{verbatim}
# text processing
# memecah kalimat menjadi kata-kata
kalimat = raw_input("Masukkan kalimat yang cukup panjang.\n")
# pisahkan menjadi kata
kata = kalimat.split()
for k in kata:
    print k
\end{verbatim}

Contoh singkat di atas menunjukkan cara memecahkan kalimat menjadi kata-kata.
Sebagai catatan, kalimat yang dimaksudkan diakhiri dengan {\em return}. 
Untuk memproses kalimat yang lebih panjang dan memiliki {\em return} harus 
dilakukan perbaikan.
Coba kembangkan program yang dapat menerima masukan dari sebuah berkas.

Dengan menggunakan ide pada bagian sebelumnya, kita dapat menghitung
jumlah kemunculan kata tertentu dalam sebuah kalimat. (Perhatikan bahwa
"kata" di sini bersifat {\em case sensitive}. Agar dia tidak bergantung
kepada huruf besar dan kecil, semua huruf harus diubah dahulu ke
huruf kecil.)

Program ini juga dapat menjadi basis dari sebuah sistem untuk menganalisis
sentimen seseorang di media sosial. Pikirkan algoritmanya untuk melakukan
hal tersebut.


\section{Python3}
Bagaimana caranya agar kita dapat menggunakan Python3 sebagai {\em default}
dari Python? Cara yang paling mudah adalah dengan menggunakan fitur
alias di shell (jika Anda menggunakan variasi dari UNIX).

\begin{verbatim}
    alias python=python3
\end{verbatim}

Jika Anda ingin membuat ini menjadi permanen dan Anda menggunakan {\em bash}
sebagai shell Anda, letakkan alias tersebut pada berkas ``.bashrc'' pada
{\em home directory} Anda (atau pada berkas ``.bash\_aliases''). 
Jika Anda menyimpannya di dalam berkas tersebut, maka perubahan baru akan 
terjadi jika Anda membuat sesi shell baru atau Anda logout dan login kembali.
Jika Anda ingin langsung aktif, bisa juga berkas tersebut di-source.

\begin{verbatim}
    source ~/.bashrc
\end{verbatim}
    
Untuk memasang modul-modul di Python3 dapat dilakukan dengan cara
memanggil python3 secara eksplisit. Sebagai contoh, untuk memasang
modul ``numpy'' pada (dengan) python3 adalah sebagai berikut.

\begin{verbatim}
python3 -m pip install numpy
\end{verbatim}

Sebagai catatan ada banyak cara untuk memasang modul Python, tetapi
cara di atas yang paling konsisten bagi saya.
