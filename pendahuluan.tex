\chapter{Pendahuluan}
Bahasa pemrograman Python mulai populer saat dikarenakan berbagai hal; mudah
dipelajari, tersedia dan banyak {\em library}-nya. Nanti akan kita bahas beberapa
library Python ini. Lengkapnya library ini juga yang menyebabkan
Python dipergunakan di berbagai aplikasi. Berbagai sekolah (dan
perguruan tinggi) bahkan mengajarkan Python sebagai pengantar pemrograman.

Bahasa Python tersedia untuk berbagai sistem operasi; Windows, Mac OS, dan
berbagai variasi dari UNIX (Linux, *BSD, dan seterusnya). Di dalam buku ini
saya akan menggunakan contoh-contoh yang saya gunakan di komputer saya yang
berbasis Linux Mint. Meskipun seharusnya semuanya kompatibel dengan berbagai
sistem operasi, kemungkinan ada hal-hal yang agak berbeda. Jika hal itu
terjadi, gunakan internet untuk mencari jawabannya.

\section{Instalasi}
Python dapat diperoleh secara gratis dari berbagai sumber. Sumber utamanya
adalah di situs python.org. Untuk sementara ini bagian ini saya serahkan kepada
Anda. Ada terlalu banyak perubahan sehingga bagian ini akan cepat kadaluwarsa.

Untuk sistem operasi berbasis Linux dan Mac OS, Python sudah terpasang sebagai
bawaan dari sistem operasinya. Jika Anda ingin menggunakan versi terbaru maka
Anda harus memasangnya sendiri dengan mengunduh instalasinya di python.org.

\section{Memulai}
Untuk memastikan Python berjalan, ketikkan "python" di terminal Linux Anda.
(Bagi yang menggunakan Windows, hal ini dapat dilakukan dengan menggunakan
CMD.exe.) Catatan, di sistem Linux, tanda ``dollar'' merupakan {\em prompt}
dari {\em shell} Anda. Jangan diketikkan.

\begin{verbatim}
$ python
Python 2.7.12 (default, Nov 20 2017, 18:23:56) 
[GCC 5.4.0 20160609] on linux2
Type "help", "copyright", "credits" or "license" for more information.
>>> 
\end{verbatim}

Dari tampilan di atas dapat kita ketahui bahwa Python yang saya gunakan adalah
versi 2.7.12. Sekarang kita dapat memulai pemrograman Python dengan menuliskan
program ``hello world'' (yang merupakan standar bagi belajar pemrograman).
Ketikkan ``print ...'' (dan seterusnya seperti di bawah ini).

\begin{verbatim}
print "Hello, world!"
Hello, world!
\end{verbatim}

Python akan menampilkan apapun yang ada di antara tanda petik tersebut. Hore!
Anda berhasil membuat program Python yang pertama.

Mari kita lanjutkan dengan membuat program yang lebih panjang. Program Python
dapat disimpan di dalam sebuah berkas untuk kemudian dieksekusi belakangan.
Buka editor kesukaan Anda dan ketikkan program hello world di atas di dalam
editor Anda tersebut. Setelah itu simpan berkas tersebut dengan nama
``hello.py''. Biasanya berkas program Python ditandai dengan akhiran
(extension) ``.py''.

Setelah berkas tersebut tersedia, maka kita dapat menjalankan Python dengan
memberikan perintah python dan nama berkas tersebut.

\begin{verbatim}
$ python hello.py
Hello, world!
\end{verbatim}

