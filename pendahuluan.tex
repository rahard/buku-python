\chapter{Pendahuluan}
Bahasa pemrograman Python mulai populer saat dikarenakan berbagai hal; mudah
dipelajari, tersedia dan banyak {\em library}-nya. Nanti akan kita bahas beberapa
library Python ini. Lengkapnya library ini juga yang menyebabkan
Python dipergunakan di berbagai aplikasi. Berbagai sekolah (dan
perguruan tinggi) bahkan mengajarkan Python sebagai pengantar pemrograman.

Bahasa Python tersedia untuk berbagai sistem operasi; Windows, Mac OS, dan
berbagai variasi dari UNIX (Linux, *BSD, dan seterusnya). Di dalam buku ini
saya akan menggunakan contoh-contoh yang saya gunakan di komputer saya yang
berbasis Linux Mint. Meskipun seharusnya semuanya kompatibel dengan berbagai
sistem operasi, kemungkinan ada hal-hal yang agak berbeda. Jika hal itu
terjadi, gunakan internet untuk mencari jawabannya.

\section{Instalasi}
Python dapat diperoleh secara gratis dari berbagai sumber. Sumber utamanya
adalah di situs python.org. Untuk sementara ini bagian ini saya serahkan kepada
Anda. Ada terlalu banyak perubahan sehingga bagian ini akan cepat kadaluwarsa.

Untuk sistem operasi berbasis Linux dan Mac OS, Python sudah terpasang sebagai
bawaan dari sistem operasinya. Jika Anda ingin menggunakan versi terbaru maka
Anda harus memasangnya sendiri dengan mengunduh instalasinya di python.org.

\section{Memulai}
Untuk memastikan Python berjalan, ketikkan "python" di terminal Linux Anda.
(Bagi yang menggunakan Windows, hal ini dapat dilakukan dengan menggunakan
CMD.exe.) Catatan, di sistem Linux, tanda ``dollar'' merupakan {\em prompt}
dari {\em shell} Anda. Jangan diketikkan.

\begin{verbatim}
$ python
Python 2.7.12 (default, Nov 20 2017, 18:23:56) 
[GCC 5.4.0 20160609] on linux2
Type "help", "copyright", "credits" or "license" for more information.
>>> 
\end{verbatim}

Dari tampilan di atas dapat kita ketahui bahwa Python yang saya gunakan adalah
versi 2.7.12. Sekarang kita dapat memulai pemrograman Python dengan menuliskan
program ``hello world'' (yang merupakan standar bagi belajar pemrograman).
Ketikkan ``print ...'' (dan seterusnya seperti di bawah ini).

\begin{verbatim}
print "Hello, world!"
Hello, world!
\end{verbatim}

Python akan menampilkan apapun yang ada di antara tanda petik tersebut. Hore!
Anda berhasil membuat program Python yang pertama.

Mari kita lanjutkan dengan membuat program yang lebih panjang. Program Python
dapat disimpan di dalam sebuah berkas untuk kemudian dieksekusi belakangan.
Buka editor kesukaan Anda dan ketikkan program hello world di atas di dalam
editor Anda tersebut. Setelah itu simpan berkas tersebut dengan nama
``hello.py''. Biasanya berkas program Python ditandai dengan akhiran
(extension) ``.py''.

Setelah berkas tersebut tersedia, maka kita dapat menjalankan Python dengan
memberikan perintah python dan nama berkas tersebut. 
Kata ``Hello world'' akan ditampilkan.

\begin{verbatim}
$ python hello.py
Hello, world!
\end{verbatim}

\section{Bahasa Python} 
Tentang bahasa Python itu sendiri akan diperdalam pada versi berikutnya.
Sementara itu fitur tentang bahasa Python akan dibahas sambil berjalan.
Pendekatan ini saya ambil untuk membuat buku menjadi lebih menarik dan lebih
singkat. Belajar seperlunya.

Hal yang sangat berbeda dari bahasa Python dengan
bahasa pemrograman lainnya adalah masalah {\em block} dari kode. Bahasa pemrograman C misalnya menggunakan tanda kurung kurawal ``\{'' untuk menyatakan blok. Sementara itu Python menggunakan {\em indentation} untuk menyatakan satu blok. Lihat contoh di bawah ini.

\begin{verbatim}
for in in range(10):
    print i
\end{verbatim}

Disarankan untuk menggunakan spasi sebanyak empat (4) buah untuk {\em
indentation} tersebut. (Ini membuat banyak perdebatan karena ada banyak orang
yang menggunakan tab bukan spasi.)

\section{Input}
Salah satu cara untuk mendapatkan masukan (input) dari pengguna secara
interaktif adalah dengan menggunakan fungsi "input" (untuk Python 3.*)
atau "raw\_input" (untuk Python 2.7).

\begin{verbatim}
# ini untuk Python 2.7
# gunakan raw_input
nama = raw_input("Masukkan nama Anda: ")
print "Selamat pagi,", nama
\end{verbatim}

Perhatikan bahwa kita menggunakan variabel "nama" untuk menyimpan masukan
dari pengguna. Variabel "nama" tersebut mempunyai tipe {\em string}.
Python mengenali secara otomatis.

Mari kita coba tampilkan huruf-huruf yang ada di dalam variabel "nama"
tersebut.

\begin{verbatim}
# for loop bisa menggunakan elemen dari string
# tidak harus indeks angka
for i in nama:
    print i
\end{verbatim}

Kita juga dapat membuat statistik kemunculan huruf dari nama (atau teks)
yang dimasukkan oleh pengguna. Statistik ini dapat dimanfaatkan untuk
proses enkripsi, misalnya. Gunakan program "input" di atas, dan gabungkan
dengan kode berikut ini.

\begin{verbatim}
# associative array: hitung jumlah huruf dan spasi
huruf = {}  # inisialisasi
for key in nama:
    if key in huruf:
        huruf[key] += 1
    else:
        huruf[key]=1
# tampilkan hasil python 2.7
# sorted() agar key-nya diurutkan
# for python 3.* use this: for key, value in d.items():
for key, value in sorted(huruf.iteritems()):
    print key, value
\end{verbatim}

Contoh program di atas menggunakan {\em associative array} atau dalam
Python disebut {\em dictionary}. Pada prinsipnya ini adalah array tetapi
dengan menggunakan {\em immutable object} seperti {\em string} sebagai
indeks atau kuncinya.

Pada contoh tersebut, spasi ({\em space}) masih dianggap sebagai huruf.
Coba ubah sehingga spasi tidak dimasukkan sebagai indeks.


\section{Pemrosesan Teks}
Salah satu manfaat utama dari bahasa pemrograman seperti Python adalah
kemampuannya dalam memproses teks ({\em text processing}). Bahasa
pemrograman lainnya, seperti C, tentu saja dapat digunakan untuk melakukan
pemrosesan teks. Namun bahasa C lebih "sulit" digunakan karena ada banyak
hal yang harus kita ketahui dari awal.

\begin{verbatim}
# text processing
# memecah kalimat menjadi kata-kata
kalimat = raw_input("Masukkan kalimat yang cukup panjang.\n")
# pisahkan menjadi kata
kata = kalimat.split()
for k in kata:
    print k
\end{verbatim}

Contoh singkat di atas menunjukkan cara memecahkan kalimat menjadi kata-kata.
Sebagai catatan, kalimat yang dimaksudkan diakhiri dengan {\em return}. 
Untuk memproses kalimat yang lebih panjang dan memiliki {\em return} harus 
dilakukan perbaikan.
Coba kembangkan program yang dapat menerima masukan dari sebuah berkas.

Dengan menggunakan ide pada bagian sebelumnya, kita dapat menghitung
jumlah kemunculan kata tertentu dalam sebuah kalimat. (Perhatikan bahwa
"kata" di sini bersifat {\em case sensitive}. Agar dia tidak bergantung
kepada huruf besar dan kecil, semua huruf harus diubah dahulu ke
huruf kecil.)

Program ini juga dapat menjadi basis dari sebuah sistem untuk menganalisis
sentimen seseorang di media sosial. Pikirkan algoritmanya untuk melakukan
hal tersebut.


\section{Python3}
Bagaimana caranya agar kita dapat menggunakan Python3 sebagai {\em default}
dari Python? Cara yang paling mudah adalah dengan menggunakan fitur
alias di shell (jika Anda menggunakan variasi dari UNIX).

\begin{verbatim}
    alias python=python3
\end{verbatim}

Jika Anda ingin membuat ini menjadi permanen dan Anda menggunakan {\em bash}
sebagai shell Anda, letakkan alias tersebut pada berkas ``.bashrc'' pada
{\em home directory} Anda (atau pada berkas ``.bash\_aliases''). 
Jika Anda menyimpannya di dalam berkas tersebut, maka perubahan baru akan 
terjadi jika Anda membuat sesi shell baru atau Anda logout dan login kembali.
Jika Anda ingin langsung aktif, bisa juga berkas tersebut di-source.

\begin{verbatim}
    source ~/.bashrc
\end{verbatim}
    
Untuk memasang modul-modul di Python3 dapat dilakukan dengan cara
memanggil python3 secara eksplisit. Sebagai contoh, untuk memasang
modul ``numpy'' pada (dengan) python3 adalah sebagai berikut.

\begin{verbatim}
python3 -m pip install numpy
\end{verbatim}