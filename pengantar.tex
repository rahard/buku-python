\chapter*{Pengantar}
Buku ini sebetulnya merupakan catatan pribadi saya dalam belajar pemrograman
dengan menggunakan bahasa Python. Saya sudah mengenal Python sejak dari jaman
dahulu kala, tetapi pada masa itu saya tidak terlalu tertarik karena saya lebih
suka menggunakan bahasa Perl. Sampai sekarang sebetulnya saya masih suka
menggunakan bahasa Perl, tetapi karena tuntutan zaman yang banyak membutuhkan
pemrograman dengan menggunakan bahasa Python maka saya kembali mempelajari
bahasa Python.

Buku ini lebih banyak menampilkan contoh-contoh yang saya gunakan untuk
mengingat-ingat hal-hal yang pernah saya kerjakan atau untuk mencari ide ketika
memecahkan masalah lain. Jadinya buku ini seperti sebuah {\em cookbook}. Semoga
pendekatan seperti ini cocok juga untuk Anda.

Yang namanya catatan tentu saja sesuai dengan apa yang saya lakukan. Basis saya
menggunakan Linux. Jadi ada kemungkinan contoh yang tidak persis sama. Demikian
pula cara saya menggunakan (memprogram dengan) Python mungkin bukan cara yang
paling sempurna, tetapi mengikuti cara saya. (Apapun itu.)

\st{Ketika buku ini ditulis, versi Python yang paling stabil adalah versi 2.7
meskipun versi 3 juga sudah banyak digunakan orang. Ada banyak bagian di dalam
buku ini yang dituliskan untuk Python versi 2.7 kemungkinan harus disesuaikan
untuk versi 3. Sebagai contoh, print harus menggunakan tanda kurung.
Namun secara prinsip mestinya sebagian besar akan tetap sama.} 

Versi buku ini sudah menganjurkan untuk lebih condong ke Python versi 3
karena pengembangan versi 2 sudah mulai dihentikan.
[Catatan: buku mulai ditulis akhir 2017.]
Bagian-bagian yang sudah terlanjur ditulis dengan menggunakan versi 2,
sedikit demi sedikit mulai dikonversikan ke versi 3.


Selamat menikmati versi 0.9\\
Bandung, Maret 2022 \\
Budi Rahardjo
